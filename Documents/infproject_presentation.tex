\documentclass[10pt]{beamer}
\usepackage[T1,T2A]{fontenc}
\usepackage[utf8]{inputenc}
\usepackage{hyperref}
\hypersetup{unicode=true}
\usepackage{fontawesome}
\usepackage{graphicx}
\usepackage[english,russian]{babel}

\usepackage[T1]{fontenc}
\usepackage{fontawesome}
\usepackage{PTSans} 
\mode<presentation>
{
  \usetheme[progressbar=foot,numbering=fraction,background=light]{metropolis} 
  \usecolortheme{default}
  \usefonttheme{default}
  \setbeamertemplate{navigation symbols}{}
  \setbeamertemplate{caption}[numbered]
} 

\let\textttorig\texttt
\renewcommand<>{\texttt}[1]{%
  \only#2{\textttorig{#1}}%
}

\usepackage{minted}

\usepackage{xcolor}
\definecolor{codecolor}{HTML}{FFC300}

\usepackage{tcolorbox}
\tcbuselibrary{most,listingsutf8,minted}

\tcbset{tcbox width=auto,left=1mm,top=1mm,bottom=1mm,
right=1mm,boxsep=1mm,middle=1pt}

\newtcblisting{myr}[1]{colback=codecolor!5,colframe=codecolor!80!black,listing only, 
minted options={numbers=left, style=tcblatex,fontsize=\tiny,breaklines,autogobble,linenos,numbersep=3mm},
left=5mm,enhanced,
title=#1, fonttitle=\bfseries,
listing engine=minted,minted language=r}

\definecolor{mygreen}{HTML}{37980D}
\definecolor{myblue}{HTML}{0D089F}
\definecolor{myred}{HTML}{98290D}

\usepackage{listings}

\lstdefinelanguage{XML}
{
  morestring=[b]",
  morecomment=[s]{<!--}{-->},
  morestring=[s]{>}{<},
  morekeywords={ref,xmlns,version,type,canonicalRef,metr,real,target}
}

\lstdefinestyle{myxml}{
language=XML,
showspaces=false,
showtabs=false,
basicstyle=\ttfamily,
columns=fullflexible,
breaklines=true,
showstringspaces=false,
breakatwhitespace=true,
escapeinside={(*@}{@*)},
basicstyle=\color{mygreen}\ttfamily,
stringstyle=\color{myred},
commentstyle=\color{myblue}\upshape,
keywordstyle=\color{myblue}\bfseries,
}


% ------------------------------------------------------------------------------
% The Document
% ------------------------------------------------------------------------------
\title{Разработка приложения <<Magic piano>>}
\subtitle{Отчет о проектной работе по курсу <<Основы информатики и программирования>>}
\author{Топчий Евгений Дмитриевич}
\date{09 июня 2021}

\begin{document}

\maketitle

\begin{frame}[fragile]{Введение}
В наше время существует множество различных игр, как компьютерных, так и мобильных, нацеленых на так назывемое <<Убийство времени>>.Я решил реализовать собственную игру такого типа, главная задача в которой набрать как можно больше очков.
\end{frame}

\begin{frame}{Требования к приложению}
Основные требования от данного приложения с точки зрения пользователя: \\
--Понятное и удобное управление \\
--Приятный дизайн \\
--Комфортная игра
\end{frame}

\begin{frame}{Проектирование приложения}
Модули приложения:
\begin{enumerate}
    \item main.cpp - главный модуль для работы с функциями, реализованными на языке <<C++>>.
    \item helper.cpp - модуль, в котором реализованы функции, связанные со счётчиком.
    \item Gameplay.qml - модуль, в котором реализован геймплей данного приложения.
    \item Info.qml - модуль, содержащий в себе информацию об управлении в реализуемой игре.
    \item Toolbar.qml - модуль, содержащий в себе кнопки <<Новая игра>>, <<Выйти>>, а так же содержащий в себе счетчик очков и счетчик комбо.
    \item main.qml - главный модуль графического интерфейса.
\end{enumerate}
\end{frame}


\begin{frame}[fragile]{Реализация приложения}
Для реализации были использованы языки <<C++>> и <<QML>>\\
Количество строк кода, написанных на <<C++>> - 60 \\
Количество строк кода, написанных на <<QML>> - 339 \\
Количество модулей - 6
\end{frame}

\begin{frame}[fragile]{Скриншоты игры}
\begin{figure}
\includegraphics[width=0.4\textwidth]{gamemoment.jpg}
\includegraphics[width=0.4\textwidth]{defeat.jpg}
\caption{Слева - игровой момент, справа - пример поражения}
\end{figure}
\end{frame}

\begin{frame}[fragile]{Заключение}
В результате работы над проектом было реализовано приложение, реализующее игру <<Magic piano>>, цель которой - набрать как можно больше очков.\\
\\
   В результате данной работы был получен опыт работы с библиотеками <<C++>>, а так же опыт работы с <<QtQuick>>.
\end{frame}

\begin{frame}[standout]
    Спасибо за внимание ~\alert{!}~
\end{frame}

\end{document}
