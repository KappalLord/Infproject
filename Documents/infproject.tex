\documentclass[a4paper,12pt]{article}

\usepackage{ucs}
\usepackage[utf8x]{inputenc}
\usepackage[russian]{babel}
%\usepackage{cmlgc}
\usepackage{graphicx}
\usepackage{listings}
\usepackage{xcolor}
\usepackage{titlesec}
%\usepackage{courier}

\makeatletter
\makeatother

\newcommand{\myrule}[1]{\rule{#1}{0.4pt}}
\newcommand{\sign}[2][~]{{\small\myrule{#2}\\[-0.7em]\makebox[#2]{\it #1}}}

% Поля
\usepackage[top=20mm, left=30mm, right=10mm, bottom=20mm, nohead]{geometry}
\usepackage{indentfirst}

% Межстрочный интервал
\renewcommand{\baselinestretch}{1.50}

% ------------------------------------------------------------------------------
% minted
% ------------------------------------------------------------------------------
\usepackage{minted}


% ------------------------------------------------------------------------------
% tcolorbox / tcblisting
% ------------------------------------------------------------------------------
\usepackage{xcolor}
\definecolor{codecolor}{HTML}{FFC300}

\usepackage{tcolorbox}
\tcbuselibrary{most,listingsutf8,minted}

\tcbset{tcbox width=auto,left=1mm,top=1mm,bottom=1mm,
right=1mm,boxsep=1mm,middle=1pt}

\newtcblisting{myr}[1]{colback=codecolor!5,colframe=codecolor!80!black,listing only, 
minted options={numbers=left, style=tcblatex,fontsize=\tiny,breaklines,autogobble,linenos,numbersep=3mm},
left=5mm,enhanced,
title=#1, fonttitle=\bfseries,
listing engine=minted,minted language=r}

%%%%%%%%%%%%%%%%%%%%%%%%%%%%%%%%%%%%%%%

\begin{document}

%%%%%%%%%%%%%%%%%%%%%%%%%%%%%%%
%%%                         %%%
%%% Начало титульного листа %%%

\thispagestyle{empty}
\begin{center}


\renewcommand{\baselinestretch}{1}
{\large
{\sc Петрозаводский государственный университет\\
Институт математики и информационных технологий\\
Кафедра информатики и математического обеспечения
}
}

\end{center}


\begin{center}
%%%%%%%%%%%%%%%%%%%%%%%%%
%
% Раскомментируйте (уберите знак процента в начале строки)
% для одной из строк типа направления  - бакалавриат/
% магистратура и для одной из
% строк Вашего направление подготовки
%
09.03.04 --- Программная инженерия  \\
% 01.03.02 --- Прикладная математика и информатика \\
% 09.03.02 --- Информационные системы и технологии \\
% 09.03.04 --- Программная инженерия \\
%%%%%%%%%%%%%%%%%%%%%%%%%
\end{center}

\vfill

\begin{center}
{\normalsize 
	Отчет о проектной работе по курсу <<Основы информатики и программирования>>}

\medskip

%%% Название работы %%%
	{\Large \sc {Приложение \\ <<Magic piano>> }} \\
\end{center}

\medskip

\begin{flushright}
\parbox{11cm}{%
\renewcommand{\baselinestretch}{1.2}
\normalsize
	Выполнил:\\
% Выполнили:\\
%%% ФИО студента %%%
студент 1 курса группы 22107
\begin{flushright}
	Е. Д. Топчий \sign[подпись]{4cm}
\end{flushright}

%%% Второй участник %%%
% студента 1 курса группы 2210X
% \begin{flushright}
% 	И. О. Фамилия \sign[подпись]{4cm}
% \end{flushright}

%%%%%%%%%%%%%%%%%%%%%%%%%
% девушкам применять "Выполнила" и "студентка"
%%%%%%%%%%%%%%%%%%%%%%%%%
}
\end{flushright}

\vfill

\begin{center}
\large
    Петрозаводск --- 2021
\end{center}

%%% Конец титульного листа  %%%
%%%                         %%%
%%%%%%%%%%%%%%%%%%%%%%%%%%%%%%%

%%%%%%%%%%%%%%%%%%%%%%%%%%%%%%%%
%%%                          %%%
%%% Содержание               %%%

\newpage

\tableofcontents

%%% Содержание              %%%
%%%                         %%%
%%%%%%%%%%%%%%%%%%%%%%%%%%%%%%%

%%%%%%%%%%%%%%%%%%%%%%%%%%%%%%%%
%%%                          %%%
%%% Введение                 %%%

%%% В введении Вы должны описать предметную область, с которой связана %%%
%%% Ваша работа, показать её актуальность, вкратце определить цель     %%%
%%% разработки					       %%%


\newpage
\section*{Введение}
\addcontentsline{toc}{section}{Введение}

Цель проекта: Разработка проиложения, реализующего игру <<Magic piano>>

Задачи проекта: 
%%% Пример создания списков %%%
\begin{enumerate} 
    \item Реализовать движение объектов;
    \item Реализовать управление путем нажатия определенных клавиш;
    \item Реализовать счетчик очков
\end{enumerate}
    \item В наше время существует множество различных игр, как компьютерных, так и мобильных, нацеленых на так назывемое <<Убийство времени>>.Я решил реализовать собственную игру такого типа, главная задача в которой набрать как можно больше очков.


%%%                          %%%
%%%%%%%%%%%%%%%%%%%%%%%%%%%%%%%%

%%%%%%%%%%%%%%%%%%%%%%%%%%%%%%%
%%% Требования к приложению %%%
\newpage
\section{Требования к приложению}
% \subsection{Подраздел}
Основные требования от данного приложения с точки зрения пользователя: \\
--Понятное и удобное управление \\
--Приятный дизайн \\
--Комфортная игра

 
%%%                                     %%%
%%%%%%%%%%%%%%%%%%%%%%%%%%%%%%%%%%%%%%%%%%%

%%%%%%%%%%%%%%%%%%%%%%%%%%%%%%%%%%%%%%%%%%%
%%%                                     %%%
%%% Проектирование приложения           %%%
\newpage
\section{Проектирование приложения}

Модули приложения:
\begin{enumerate}
    \item main.cpp - главный модуль для работы с функциями, реализованными на языке <<C++>>.
    \item helper.cpp - модуль, в котором реализованы функции, связанные со счётчиком.
    \item Gameplay.qml - модуль, в котором реализован геймплей данного приложения.
    \item Info.qml - модуль, содержащий в себе информацию об управлении в реализуемой игре.
    \item Toolbar.qml - модуль, содержащий в себе кнопки <<Новая игра>>, <<Выйти>>, а так же содержащий в себе счетчик очков и счетчик комбо.
    \item main.qml - главный модуль графического интерфейса.
\end{enumerate}

%%%                          %%%
%%%%%%%%%%%%%%%%%%%%%%%%%%%%%%%%

%%%%%%%%%%%%%%%%%%%%%%%%%%%%%%%%
%%%                          %%%
%%% Реализация приложения    %%%
\newpage
\section{Реализация приложения}

Для реализации были использованы языки <<C++>> и <<QML>>\\
Количество строк кода, написанных на <<C++>> - 60 \\
Количество строк кода, написанных на <<QML>> - 339 \\
Количество модулей - 6


%%% Если необходимо вставками оформляются исключительно небольшие фрагменты кода.
%%% Для больших фрагментов используте приложение (пример после заключения)




%%%                          %%%
%%%%%%%%%%%%%%%%%%%%%%%%%%%%%%%%

%%%%%%%%%%%%%%%%%%%%%%%%%%%%%%%%
%%%                          %%%
%%% Заключение               %%%

\newpage
\section*{Заключение}
\addcontentsline{toc}{section}{Заключение}
В результате работы над проектом было реализовано приложение, реализующее игру <<Magic piano>>, цель которой - набрать как можно больше очков.\\
\\
   В результате данной работы был получен опыт работы с библиотеками <<C++>>, а так же опыт работы с <<QtQuick>>.

%%%                          %%%
%%%%%%%%%%%%%%%%%%%%%%%%%%%%%%%%

%%%%%%%%%%%%%%%%%%%%%%%%%%%%%%%%
%%%                          %%%
%%% Приложение               %%%

\newpage
\appendix
%\section*{Приложение}
%\addcontentsline{toc}{section}{Приложение}
%\titleformat{\section}[display]
%  {\normalfont\Large\bfseries}
%  {Приложение\ \thesection}
%  {0pt}{\Large\centering}
%\renewcommand{\thesection}{\Asbuk{section}}

\end{document}
